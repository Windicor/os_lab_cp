\section{Общий метод и алгоритм решения}

Проект будет состоять из сервера и клиета, собираться при помощи утилиты make. Работать будет на основе сервера сообщений ZMQ и сокетов ipc. Для работы с сервером сообщений буду использовать код, написанный в процессе работы над лабораторной 6-8. Клиент работает в двух потоках: первый - взаимодействие с пользователем и отправка сообщений, второй - получение сообщений. Сервер работает в одном потоке, реагируя на поступающие сообщения.

При запуске клиент дожидается ответа от сервера, затем устанавливается специально выделенная пара сокетов для взаимодействия сервера с конкретным клиентом. Каждому клиенту присваевается временный идентификатор. Затем пользователь либо регистрируется, либо входит по зарегистрированной паре логин-пароль. Затем пользователь выбирает, с кем хочет соединиться для переписки (по никнейму). В случае успеха появляется возможность обмена сообщениями и файлами.

Текстовые сообщения ограничены в длине 1000-ей знаков. Файлы отправляются пакетами по 1 Мб. Логин и пароль хранятся в зашифрованном виде.

\pagebreak

\section{Исходный код}

\textbf{m\_zmq.h}

\begin{lstlisting}[language=C++]

#pragma once

#include <memory>
#include <string>

#include "message.h"

void* create_zmq_context();
void destroy_zmq_context(void* context);

enum class SocketType {
  PUBLISHER,
  SUBSCRIBER
};
void* create_zmq_socket(void* context, SocketType type);
void close_zmq_socket(void* socket);

const std::string ENDPOINT_PROTOCOL = "ipc://";
const std::string ENDPOINT_FOLDER = "tmp/";
enum class EndpointType {
  SERVER_PUB_GENERAL,
  SERVER_SUB_GENERAL,
  SERVER_PUB,
  CLIENT_PUB
};
std::string create_endpoint(EndpointType type, int id = 0);

void bind_zmq_socket(void* socket, std::string endpoint);
void unbind_zmq_socket(void* socket, std::string endpoint);
void connect_zmq_socket(void* socket, std::string endpoint);
void disconnect_zmq_socket(void* socket, std::string endpoint);

void send_zmq_msg(void* socket, std::shared_ptr<Message> msg);
std::shared_ptr<Message> get_zmq_msg(void* socket);

\end{lstlisting}

\textbf{m\_zmq.cpp}

\begin{lstlisting}[language=C++]

#include "m_zmq.h"

#include <unistd.h>
#include <zmq.h>

#include <algorithm>
#include <cstring>
#include <iostream>

using namespace std;

void* create_zmq_context() {
  void* context = zmq_ctx_new();
  if (context == NULL) {
    throw runtime_error("Can't create new context. pid:" + to_string(getpid()));
  }
  return context;
}

void destroy_zmq_context(void* context) {
  if (zmq_ctx_destroy(context) != 0) {
    throw runtime_error("Can't destroy context. pid:" + to_string(getpid()));
  }
}

int get_zmq_socket_type(SocketType type) {
  switch (type) {
    case SocketType::PUBLISHER:
      return ZMQ_PUB;
    case SocketType::SUBSCRIBER:
      return ZMQ_SUB;
    default:
      throw logic_error("Undefined socket type");
  }
}

void* create_zmq_socket(void* context, SocketType type) {
  int zmq_type = get_zmq_socket_type(type);
  void* socket = zmq_socket(context, zmq_type);
  if (socket == NULL) {
    throw runtime_error("Can't create socket");
  }
  if (zmq_type == ZMQ_SUB) {
    if (zmq_setsockopt(socket, ZMQ_SUBSCRIBE, 0, 0) == -1) {
      throw runtime_error("Can't set ZMQ_SUBSCRIBE option");
    }
  }
  int linger_period = 0;
  if (zmq_setsockopt(socket, ZMQ_LINGER, &linger_period, sizeof(int)) == -1) {
    throw runtime_error("Can't set ZMQ_LINGER option");
  }
  return socket;
}

void close_zmq_socket(void* socket) {
  sleep(1);  // Don't comment it, because sometimes zmq_close blocks
  if (zmq_close(socket) != 0) {
    throw runtime_error("Can't close socket");
  }
}

string create_endpoint(EndpointType type, int id) {
  switch (type) {
    case EndpointType::SERVER_PUB_GENERAL:
      return ENDPOINT_PROTOCOL + ENDPOINT_FOLDER + "server_pub_general"s;
    case EndpointType::SERVER_SUB_GENERAL:
      return ENDPOINT_PROTOCOL + ENDPOINT_FOLDER + "server_sub_general"s;
    case EndpointType::SERVER_PUB:
      return ENDPOINT_PROTOCOL + ENDPOINT_FOLDER + "server_pub_"s + to_string(id);
    case EndpointType::CLIENT_PUB:
      return ENDPOINT_PROTOCOL + ENDPOINT_FOLDER + "client_pub_"s + to_string(id);
    default:
      throw logic_error("Undefined endpoint type");
  }
}

void bind_zmq_socket(void* socket, string endpoint) {
  if (zmq_bind(socket, endpoint.data()) != 0) {
    throw runtime_error("Can't bind socket (create 'tmp' folder)");
  }
}

void unbind_zmq_socket(void* socket, string endpoint) {
  if (zmq_unbind(socket, endpoint.data()) != 0) {
    throw runtime_error("Can't unbind socket");
  }
}

void connect_zmq_socket(void* socket, string endpoint) {
  if (zmq_connect(socket, endpoint.data()) != 0) {
    throw runtime_error("Can't connect socket (create 'tmp' folder)");
  }
}

void disconnect_zmq_socket(void* socket, string endpoint) {
  if (zmq_disconnect(socket, endpoint.data()) != 0) {
    throw runtime_error("Can't disconnect socket");
  }
}

void create_zmq_msg(zmq_msg_t* zmq_msg, shared_ptr<Message> msg_ptr) {
  switch (msg_ptr->type()) {
    case MessageType::BASIC:
      zmq_msg_init_size(zmq_msg, sizeof(Message));
      *(Message*)zmq_msg_data(zmq_msg) = *(Message*)msg_ptr.get();
      break;
    case MessageType::TEXT:
      zmq_msg_init_size(zmq_msg, sizeof(TextMessage));
      *(TextMessage*)zmq_msg_data(zmq_msg) = *(TextMessage*)msg_ptr.get();
      break;
    case MessageType::FILE:
      zmq_msg_init_size(zmq_msg, sizeof(FileMessage));
      *(FileMessage*)zmq_msg_data(zmq_msg) = *(FileMessage*)msg_ptr.get();
      break;
    default:
      throw logic_error("Unemplemented message type");
  }
}

void send_zmq_msg(void* socket, shared_ptr<Message> msg_ptr) {
  zmq_msg_t zmq_msg;
  create_zmq_msg(&zmq_msg, msg_ptr);
  if (!zmq_msg_send(&zmq_msg, socket, 0)) {
    throw runtime_error("Can't send message");
  }
  zmq_msg_close(&zmq_msg);
}

shared_ptr<Message> get_zmq_msg(void* socket) {
  zmq_msg_t zmq_msg;
  zmq_msg_init(&zmq_msg);
  if (zmq_msg_recv(&zmq_msg, socket, 0) == -1) {
    return Message::error_message();
  }
  shared_ptr<Message> msg_ptr = make_shared<Message>(*(Message*)zmq_msg_data(&zmq_msg));

  switch (msg_ptr->type()) {
    case MessageType::BASIC:
      break;
    case MessageType::TEXT:
      msg_ptr = make_shared<TextMessage>(*(TextMessage*)zmq_msg_data(&zmq_msg));
      break;
    case MessageType::FILE:
      msg_ptr = make_shared<FileMessage>(*(FileMessage*)zmq_msg_data(&zmq_msg));
      break;
    default:
      throw logic_error("Unemplemented message type");
  }

  zmq_msg_close(&zmq_msg);
  return msg_ptr;
}

\end{lstlisting}

\textbf{message.h}

\begin{lstlisting}[language=C++]

#pragma once

#include <cstdint>
#include <memory>
#include <string>
#include <vector>

enum class MessageType {
  BASIC,
  TEXT,
  FILE
};

enum class CommandType {
  ERROR,
  CONNECT,
  DISCONNECT,
  TEXT,
  REGISTER,
  LOGIN,
  CREATE_CHAT,
  LEFT_CHAT,
  FILE_NAME,
  FILE_PART
};

class Message {
 protected:
  MessageType type_ = MessageType::BASIC;

 public:
  CommandType command = CommandType::ERROR;
  int from_id = 0;
  int to_id = 0;
  int value = 0;

  Message() = default;
  Message(CommandType command, int from_id, int to_id, int value);
  virtual ~Message() = default;

  std::string get_stats() const;
  MessageType type() const {
    return type_;
  }

  static std::shared_ptr<Message> error_message();
  static std::shared_ptr<Message> connect_message(int id);
  static std::shared_ptr<Message> disconnect_message(int id);
  static std::shared_ptr<Message> left_chat_message(int id);
};

class TextMessage : public Message {
 public:
  static const size_t MAX_MESSAGE_SIZE = 1024;

  char text[MAX_MESSAGE_SIZE + 1];

  TextMessage();
  TextMessage(CommandType command, int from_id, int to_id, const std::string& text, int value = 0);
};

class FileMessage : public Message {
 public:
  static const int COMMON_PART = 0;
  static const int LAST_PART = 1;

  static const size_t BUF_SIZE = 1000000;

  uint8_t buf[BUF_SIZE];
  size_t size;

  FileMessage(CommandType command, int from_id, int to_id, int value, const std::vector<uint8_t>& buf_vec);
};

\end{lstlisting}

\textbf{message.cpp}

\begin{lstlisting}[language=C++]

#include "message.h"

#include <unistd.h>

#include <cstring>
#include <stdexcept>

using namespace std;

Message::Message(CommandType command, int from_id, int to_id, int value)
    : command(command),
      from_id(from_id),
      to_id(to_id),
      value(value) {
}

std::string Message::get_stats() const {
  return to_string(static_cast<int>(command)) + " " + to_string(from_id) + " " + to_string(to_id) + " " + to_string(value);
}
shared_ptr<Message> Message::error_message() {
  return make_shared<Message>(CommandType::ERROR, 0, 0, 0);
}
shared_ptr<Message> Message::connect_message(int id) {
  return make_shared<Message>(CommandType::CONNECT, id, 0, 0);
}
shared_ptr<Message> Message::disconnect_message(int id) {
  return make_shared<Message>(CommandType::DISCONNECT, id, 0, 0);
}
shared_ptr<Message> Message::left_chat_message(int id) {
  return make_shared<Message>(CommandType::LEFT_CHAT, id, 0, 0);
}

TextMessage::TextMessage() {
  type_ = MessageType::TEXT;
  text[0] = '\0';
}
TextMessage::TextMessage(CommandType command, int from_id, int to_id, const string& text_str, int value)
    : Message(command, from_id, to_id, value) {
  type_ = MessageType::TEXT;
  if (text_str.size() > MAX_MESSAGE_SIZE) {
    throw logic_error("Message text can't be longer, than MAX_MESSAGE_SIZE");
  }
  memcpy(text, text_str.data(), text_str.size() + 1);
}

FileMessage::FileMessage(CommandType command, int from_id, int to_id, int value, const vector<uint8_t>& buf_vec)
    : Message(command, from_id, to_id, value) {
  type_ = MessageType::FILE;
  if (buf_vec.size() > BUF_SIZE) {
    throw logic_error("File message size cannot be more than BUF_SIZE");
  }
  size = buf_vec.size();
  memcpy(buf, buf_vec.data(), buf_vec.size());
}


\end{lstlisting}

\textbf{socket.h}

\begin{lstlisting}[language=C++]

#pragma once

#include <string>

#include "m_zmq.h"

enum class ConnectionType {
  BIND,
  CONNECT
};

class Socket {
 public:
  Socket(void* context, SocketType socket_type, std::string endpoint);
  ~Socket();

  void send(std::shared_ptr<Message> message);
  std::shared_ptr<Message> receive();

  void subscribe(std::string endpoint);
  void unsubscribe(std::string endpoint);
  std::string endpoint() const;

 private:
  void* socket_;
  SocketType socket_type_;
  std::string endpoint_;
};

\end{lstlisting}

\textbf{client.h}

\begin{lstlisting}[language=C++]

#pragma once

#include <unistd.h>

#include <cstdint>
#include <filesystem>
#include <memory>
#include <string>
#include <vector>

#include "logger.h"
#include "socket.h"

class Client {
 public:
  Client();
  ~Client();

  void log(std::string message);
  void connect_to_server();
  void disconnect_from_server();

  void enter_in_system();
  void register_form();
  void login_form();

  void send_text_msg(std::string message);
  void send_file_msg(std::filesystem::path path);
  void enter_chat(std::string uname);
  void leave_chat();

  int id() const;

  enum class Status {
    UNLOGGED,
    LOGGED,
    LOG_ERROR,
    IN_CHAT
  };
  Status status = Status::UNLOGGED;

  friend void* second_thread(void* cli_arg);

 private:
  int id_ = -1;
  void* context_ = nullptr;
  std::unique_ptr<Socket> publiser_;
  std::unique_ptr<Socket> subscriber_;

  pthread_t second_thread_id_;
  bool server_is_avaible_ = false;

  bool terminated_ = false;
  Logger logger_ = Logger("log.txt");

  void send(std::shared_ptr<Message> message);
  std::shared_ptr<Message> receive();
  void send_file_part_msg(const std::vector<uint8_t>& file_part, int value);

  void leave_chat_actions();
  std::ifstream fin;
  std::ofstream fout;
};

\end{lstlisting}

\textbf{client.cpp}

\begin{lstlisting}[language=C++]

#include "client.h"

#include <cstdlib>
#include <ctime>
#include <fstream>
#include <iostream>

#include "m_zmq.h"
#include "md5sum.h"

using namespace std;

const int ERR_LOOP = 2;
const int MESSAGE_WAITING_TIME = 1;

const string FILES_FOLDER = "files/";

void* second_thread(void* cli_arg) {
  Client* client_ptr = (Client*)cli_arg;
  try {
    string endpoint = create_endpoint(EndpointType::SERVER_PUB_GENERAL);
    client_ptr->subscriber_ = make_unique<Socket>(client_ptr->context_, SocketType::SUBSCRIBER, move(endpoint));

    while (!client_ptr->terminated_) {
      shared_ptr<Message> msg_ptr = client_ptr->receive();
      if (msg_ptr->command == CommandType::ERROR) {
        if (client_ptr->terminated_) {
          return NULL;
        } else {
          cout << "Error" << endl;
          continue;
        }
      }
      if (msg_ptr->to_id != client_ptr->id()) {
        continue;
      }
      client_ptr->log("Message received by client: "s + msg_ptr->get_stats());
      switch (msg_ptr->command) {
        case CommandType::CONNECT:
          client_ptr->id_ = msg_ptr->value;
          client_ptr->server_is_avaible_ = true;

          endpoint = create_endpoint(EndpointType::CLIENT_PUB, client_ptr->id());
          client_ptr->publiser_ = nullptr;
          client_ptr->publiser_ = make_unique<Socket>(client_ptr->context_, SocketType::PUBLISHER, move(endpoint));

          endpoint = create_endpoint(EndpointType::SERVER_PUB, client_ptr->id());
          client_ptr->subscriber_ = nullptr;
          client_ptr->subscriber_ = make_unique<Socket>(client_ptr->context_, SocketType::SUBSCRIBER, move(endpoint));
          break;
        case CommandType::REGISTER:
        case CommandType::LOGIN:
          if (msg_ptr->value) {
            client_ptr->status = Client::Status::LOGGED;
          } else {
            client_ptr->status = Client::Status::LOG_ERROR;
          }
          break;
        case CommandType::CREATE_CHAT:
          if (msg_ptr->value) {
            cout << "You're in chat with: " << ((TextMessage*)msg_ptr.get())->text << endl;
            client_ptr->status = Client::Status::IN_CHAT;
          } else {
            cout << "Can't create chat" << endl;
          }
          break;
        case CommandType::LEFT_CHAT:
          cout << "Companion left the chat" << endl;
          cout << "Enter your companion username" << endl;
          client_ptr->leave_chat_actions();
          break;
        case CommandType::TEXT:
          cout << ">" << ((TextMessage*)msg_ptr.get())->text << endl;
          break;
        case CommandType::FILE_NAME: {
          if (!filesystem::exists(FILES_FOLDER)) {
            filesystem::create_directory(FILES_FOLDER);
          }
          string name = ((TextMessage*)msg_ptr.get())->text;
          cout << "Filename: " << name << endl;
          client_ptr->fout.open(FILES_FOLDER + name, ios::binary);
          if (!client_ptr->fout.is_open()) {
            throw runtime_error("Cannot open a file");
          }
          break;
        }
        case CommandType::FILE_PART: {
          FileMessage& file_msg = *(FileMessage*)msg_ptr.get();
          client_ptr->fout.write(reinterpret_cast<char*>(file_msg.buf), file_msg.size);
          if (file_msg.value == FileMessage::LAST_PART) {
            client_ptr->fout.close();
            cout << "File is received" << endl;
          }
          break;
        }
        default:
          throw logic_error("Undefined command type");
          break;
      }
    }

  } catch (exception& ex) {
    client_ptr->log("Client exctption: "s + ex.what() + "\nTerminated by exception on client receive loop"s);
    cout << "Terminated by error on loop" << endl;
    exit(ERR_LOOP);
  }
  return NULL;
}

Client::Client() {
  log("Starting client...");
  context_ = create_zmq_context();

  if (!filesystem::exists(ENDPOINT_FOLDER)) {
    filesystem::create_directory(ENDPOINT_FOLDER);
  }

  string endpoint = create_endpoint(EndpointType::SERVER_SUB_GENERAL);
  publiser_ = make_unique<Socket>(context_, SocketType::PUBLISHER, move(endpoint));

  if (pthread_create(&second_thread_id_, 0, second_thread, this) != 0) {
    cout << "Can't run second thread" << endl;
    exit(ERR_LOOP);
  }

  srand(time(NULL) + clock());
  id_ = rand();
}

Client::~Client() {
  if (terminated_) {
    log("Client double termination");
    return;
  }
  disconnect_from_server();
  log("Destroying client...");
  terminated_ = true;

  try {
    publiser_ = nullptr;
    subscriber_ = nullptr;
    destroy_zmq_context(context_);
    pthread_join(second_thread_id_, NULL);
  } catch (exception& ex) {
    log("Client wasn't destroyed: "s + ex.what());
  }
}

int Client::id() const {
  return id_;
}

void Client::send(shared_ptr<Message> message) {
  publiser_->send(message);
  log("Message sended from client: "s + message->get_stats());
}

shared_ptr<Message> Client::receive() {
  return subscriber_->receive();
}

void Client::log(string message) {
  logger_.log(move(message));
}

void Client::connect_to_server() {
  while (!server_is_avaible_) {
    cout << "Trying to connect to the server..." << endl;
    send(Message::connect_message(id_));
    sleep(MESSAGE_WAITING_TIME);
  }
  cout << "Connected to server" << endl;
}

void Client::disconnect_from_server() {
  cout << "Disconnecting from the server..." << endl;
  send(Message::disconnect_message(id_));
}

const int LOGIN_CHECK_COUNT = 5;
void Client::login_form() {
  cout << "Enter login and password" << endl;
  string uname, log, pas;
  if (!(cin >> log >> pas)) {
    throw runtime_error("Incorrect input");
  }
  status = Client::Status::UNLOGGED;
  send(make_shared<TextMessage>(CommandType::LOGIN, id_, 0, md5sum(log) + " "s + md5sum(pas)));
  int cnt = LOGIN_CHECK_COUNT;
  while (cnt-- > 0 && status == Client::Status::UNLOGGED) {
    cout << "Checking..." << endl;
    sleep(1);
  }
  if (status == Client::Status::LOG_ERROR || status == Client::Status::UNLOGGED) {
    cout << "Please, try again" << endl;
  }
}

void Client::register_form() {
  cout << "Enter username, login and password" << endl;
  string uname, log, pas;
  if (!(cin >> uname >> log >> pas)) {
    throw runtime_error("Incorrect input");
  }
  status = Client::Status::UNLOGGED;
  send(make_shared<TextMessage>(CommandType::REGISTER, id_, 0, uname + " "s + md5sum(log) + " "s + md5sum(pas)));
  int cnt = LOGIN_CHECK_COUNT;
  while (cnt-- > 0 && status == Client::Status::UNLOGGED) {
    cout << "Checking..." << endl;
    sleep(1);
  }
  if (status == Client::Status::LOG_ERROR || status == Client::Status::UNLOGGED) {
    cout << "Please, try again" << endl;
  }
}

void Client::enter_in_system() {
  while (status != Client::Status::LOGGED) {
    cout << "Do you have an account? (y/n)" << endl;
    string str;
    if (!(cin >> str)) {
      throw runtime_error("Incorrect input");
    }
    if (str == "y") {
      login_form();
    } else if (str == "n") {
      register_form();
    } else {
      cout << "Please, answer 'y' or 'n'" << endl;
    }
  }
  cout << "Autorized" << endl;
}

void Client::send_text_msg(string message) {
  send(make_shared<TextMessage>(CommandType::TEXT, id_, 0, move(message)));
}

void Client::send_file_part_msg(const vector<uint8_t>& file_part, int value) {
  send(make_shared<FileMessage>(CommandType::FILE_PART, id_, 0, value, move(file_part)));
}

void Client::send_file_msg(filesystem::path path) {
  if (!filesystem::is_regular_file(path)) {
    cout << "No such file" << endl;
    return;
  }
  send(make_shared<TextMessage>(CommandType::FILE_NAME, id_, 0, path.filename()));
  fin.open(path, ios::binary);
  if (!fin.is_open()) {
    cout << "Cannot open a file" << endl;
    return;
  }
  vector<uint8_t> vec(FileMessage::BUF_SIZE);
  size_t file_size = filesystem::file_size(path);
  while (fin) {
    fin.read(reinterpret_cast<char*>(vec.data()), FileMessage::BUF_SIZE);
    vec.resize(fin.gcount());

    if (fin) {
      send_file_part_msg(vec, FileMessage::COMMON_PART);
    } else {
      send_file_part_msg(vec, FileMessage::LAST_PART);
    }
  }
  fin.close();
  cout << "File is sent" << endl;
}

void Client::enter_chat(string uname) {
  cout << "Trying to create chat whith " << uname << "..." << endl;
  send(make_shared<TextMessage>(CommandType::CREATE_CHAT, id_, 0, move(uname)));
}

void Client::leave_chat_actions() {
  status = Client::Status::LOGGED;
}

void Client::leave_chat() {
  cout << "Exiting from chat..." << endl;
  send(Message::left_chat_message(id_));
  leave_chat_actions();
  cout << "Enter your companion username" << endl;
}


\end{lstlisting}

\textbf{client\_main.cpp}

\begin{lstlisting}[language=C++]

#include <signal.h>

#include <iostream>
#include <sstream>
#include <string>

#include "client.h"

using namespace std;

const int ERR_TERMINATED = 1;
const int UNIVERSAL_MESSAGE_ID = -256;

Client* client_ptr = nullptr;
void TerminateByUser(int) {
  if (client_ptr != nullptr) {
    client_ptr->~Client();
    client_ptr->log("Client is terminated by user");
  }
  exit(0);
}

const string EXIT_COMMAND = "\\exit";
const string FILE_COMMAND = "\\file";

int main() {
  try {
    if (signal(SIGINT, TerminateByUser) == SIG_ERR) {
      throw runtime_error("Can't set SIGINT signal");
    }
    if (signal(SIGSEGV, TerminateByUser) == SIG_ERR) {
      throw runtime_error("Can't set SIGSEGV signal");
    }

    Client client;
    client_ptr = &client;
    client.log("Client is started correctly");

    client.connect_to_server();
    client.enter_in_system();

    cout << "Enter your companion username" << endl;
    string text;
    while (getline(cin, text)) {
      if (text == "") {
        continue;
      }
      if (text.size() > TextMessage::MAX_MESSAGE_SIZE) {
        cout << "Too long message" << endl;
        continue;
      }

      if (client.status == Client::Status::IN_CHAT) {
        if (text == EXIT_COMMAND) {
          client.leave_chat();
        } else if (text.size() > FILE_COMMAND.size() + 1 && text.substr(0, FILE_COMMAND.size()) == FILE_COMMAND && text[FILE_COMMAND.size()] == ' ') {
          client.send_file_msg(text.substr(FILE_COMMAND.size() + 1));
        } else {
          client.send_text_msg(move(text));
        }
      } else {
        client.enter_chat(move(text));
      }
    }

  } catch (exception& ex) {
    cout << (to_string(getpid()) + " Client exception: "s + ex.what() + "\nClient terminated by exception"s) << endl;
    exit(ERR_TERMINATED);
  }
  return 0;
}

\end{lstlisting}

\textbf{server.h}

\begin{lstlisting}[language=C++]

#pragma once

#include <unistd.h>

#include <memory>
#include <optional>
#include <unordered_map>

#include "logger.h"
#include "security.h"
#include "socket.h"

class Online {
 public:
  void add_user(std::string username, int id);
  void remove_user(int id);
  std::optional<int> get_id(std::string username) const;
  std::optional<std::string> get_username(int id) const;
  bool check_username(std::string username) const;

 private:
  std::unordered_map<int, std::string> id_to_username_;
  std::unordered_map<std::string, int> username_to_id_;
};

class Rooms {
 public:
  bool add_room(int id1, int id2);
  void remove_room(int id);
  std::optional<int> get_companion(int id) const;
  bool check_companion(int id) const;

 private:
  std::unordered_map<int, int> rooms_;
};

class Server {
 public:
  Server();
  ~Server();

  std::shared_ptr<Message> receive();

  void log(std::string message);
  void add_connection(int id);
  void remove_connection(int id);
  void register_form(std::shared_ptr<Message> msg_ptr);
  void login_form(std::shared_ptr<Message> msg_ptr);
  void create_room(int from_id, std::string username);
  void exit_room(int id);
  void send_from_user_to_user(int from_id, std::shared_ptr<Message> message);

 private:
  void* context_ = nullptr;
  std::unique_ptr<Socket> subscriber_;
  std::unique_ptr<Socket> general_publiser_;
  std::unordered_map<int, std::unique_ptr<Socket>> id_to_publisher_;
  Online online_;
  Rooms rooms_;

  int id_cntr = 0;

  Logger logger_;
  Security security;

  void send_to_general(std::shared_ptr<Message> message);
  void send_to_user(int id, std::shared_ptr<Message> message);
};


\end{lstlisting}

\textbf{server.cpp}

\begin{lstlisting}[language=C++]

#include "server.h"

#include <pthread.h>
#include <signal.h>

#include <filesystem>
#include <iostream>
#include <sstream>

#include "m_zmq.h"

using namespace std;

void Online::add_user(std::string username, int id) {
  id_to_username_[id] = username;
  username_to_id_[move(username)] = id;
}
void Online::remove_user(int id) {
  auto it = id_to_username_.find(id);
  if (it != id_to_username_.end()) {
    string user = it->second;
    auto it2 = username_to_id_.find(move(user));
    id_to_username_.erase(it);
    username_to_id_.erase(it2);
  } else {
    cerr << "User to remove not found" << endl;
  }
}
optional<int> Online::get_id(std::string username) const {
  auto it = username_to_id_.find(move(username));
  if (it != username_to_id_.end()) {
    return it->second;
  } else {
    return nullopt;
  }
}
optional<std::string> Online::get_username(int id) const {
  auto it = id_to_username_.find(id);
  if (it != id_to_username_.end()) {
    return it->second;
  } else {
    return nullopt;
  }
}
bool Online::check_username(std::string username) const {
  auto it = username_to_id_.find(move(username));
  return it != username_to_id_.end();
}

bool Rooms::add_room(int id1, int id2) {
  if (rooms_.count(id1) > 0 || rooms_.count(id2) > 0) {
    return false;
  }
  rooms_[id1] = id2;
  rooms_[id2] = id1;
  return true;
}
void Rooms::remove_room(int id) {
  int id2 = rooms_[id];
  rooms_.erase(id);
  rooms_.erase(id2);
}
optional<int> Rooms::get_companion(int id) const {
  auto it = rooms_.find(move(id));
  if (it != rooms_.end()) {
    return it->second;
  } else {
    return nullopt;
  }
}
bool Rooms::check_companion(int id) const {
  return (bool)get_companion(id);
}

Server::Server() {
  log("Starting server...");
  context_ = create_zmq_context();

  if (!filesystem::exists(ENDPOINT_FOLDER)) {
    filesystem::create_directory(ENDPOINT_FOLDER);
  }

  string endpoint = create_endpoint(EndpointType::SERVER_PUB_GENERAL);
  general_publiser_ = make_unique<Socket>(context_, SocketType::PUBLISHER, move(endpoint));
  endpoint = create_endpoint(EndpointType::SERVER_SUB_GENERAL);
  subscriber_ = make_unique<Socket>(context_, SocketType::SUBSCRIBER, move(endpoint));
}

Server::~Server() {
  log("Destroying server...");
  try {
    general_publiser_ = nullptr;
    subscriber_ = nullptr;
    for (auto& [_, ptr] : id_to_publisher_) {
      ptr = nullptr;
    }
    destroy_zmq_context(context_);
  } catch (exception& ex) {
    log("Server wasn't destroyed: "s + ex.what());
  }
}

void Server::send_to_general(shared_ptr<Message> message) {
  general_publiser_->send(message);
  log("Message sended from server: "s + message->get_stats());
}

void Server::send_to_user(int id, std::shared_ptr<Message> message) {
  auto it = id_to_publisher_.find(id);
  if (it == id_to_publisher_.end()) {
    log("Message to id, that does not exist");
    return;
  }
  message->from_id = 0;
  message->to_id = id;
  id_to_publisher_[id]->send(message);
  log("Message sended from server to "s + to_string(id) + ": "s + message->get_stats());
}

void Server::send_from_user_to_user(int from_id, std::shared_ptr<Message> message) {
  auto opt = rooms_.get_companion(from_id);
  if (!opt) {
    send_to_user(from_id, Message::error_message());
    return;
  }
  int to_id = *opt;
  send_to_user(to_id, message);
}

shared_ptr<Message> Server::receive() {
  shared_ptr<Message> message = subscriber_->receive();
  log("Message received by server: "s + message->get_stats());
  if (message->type() == MessageType::TEXT) {
    log("Text: \""s + string(((TextMessage*)message.get())->text) + "\""s);
  }
  if (message->type() == MessageType::FILE) {
    log("Package size: "s + to_string(((FileMessage*)message.get())->size));
  }
  return message;
}

void Server::log(std::string message) {
  logger_.log(move(message));
}

void Server::add_connection(int id) {
  int new_id = ++id_cntr;
  string endpoint = create_endpoint(EndpointType::SERVER_PUB, new_id);
  id_to_publisher_[new_id] = make_unique<Socket>(context_, SocketType::PUBLISHER, move(endpoint));
  endpoint = create_endpoint(EndpointType::CLIENT_PUB, new_id);
  subscriber_->subscribe(move(endpoint));

  log("Connection added");
  send_to_general(make_shared<Message>(CommandType::CONNECT, 0, id, new_id));
}

void print_bool(bool b) {
  cout << b << endl;
}

void Server::create_room(int from_id, string username) {
  auto opt = online_.get_id(username);
  cout << *online_.get_username(from_id) << "a a" << username << "a" << endl;
  if (!opt || rooms_.check_companion(from_id) || rooms_.check_companion(*opt) || *online_.get_username(from_id) == username) {
    send_to_user(from_id, make_shared<Message>(CommandType::CREATE_CHAT, 0, from_id, 0));
    return;
  }
  int to_id = *opt;
  rooms_.add_room(from_id, to_id);
  send_to_user(from_id, make_shared<TextMessage>(CommandType::CREATE_CHAT, 0, from_id, *online_.get_username(to_id), 1));
  send_to_user(to_id, make_shared<TextMessage>(CommandType::CREATE_CHAT, 0, to_id, *online_.get_username(from_id), 1));
}

void Server::exit_room(int id) {
  auto opt = rooms_.get_companion(id);
  if (opt) {
    int companion_id = *opt;
    send_to_user(companion_id, make_shared<Message>(CommandType::LEFT_CHAT, 0, companion_id, 0));
  }
  rooms_.remove_room(id);
}

void Server::remove_connection(int id) {
  exit_room(id);
  online_.remove_user(id);
  id_to_publisher_.erase(id);
  string endpoint = create_endpoint(EndpointType::CLIENT_PUB, id);
  subscriber_->unsubscribe(move(endpoint));
  log("Connection removed");
}

void Server::register_form(std::shared_ptr<Message> msg_ptr) {
  TextMessage* text_msg_ptr = (TextMessage*)msg_ptr.get();
  istringstream iss(text_msg_ptr->text);
  string uname;
  LogAndPas lap;
  iss >> uname >> lap;
  if (security.Register(uname, move(lap))) {
    send_to_user(msg_ptr->from_id, make_shared<Message>(CommandType::REGISTER, 0, msg_ptr->from_id, 1));
    online_.add_user(uname, msg_ptr->from_id);
  } else {
    send_to_user(msg_ptr->from_id, make_shared<Message>(CommandType::REGISTER, 0, msg_ptr->from_id, 0));
  }
}

void Server::login_form(std::shared_ptr<Message> msg_ptr) {
  TextMessage* text_msg_ptr = (TextMessage*)msg_ptr.get();
  istringstream iss(text_msg_ptr->text);
  LogAndPas lap;
  iss >> lap;

  auto opt_uname = security.get_username(move(lap));
  if (opt_uname && !online_.check_username(*opt_uname)) {
    send_to_user(msg_ptr->from_id, make_shared<Message>(CommandType::LOGIN, 0, msg_ptr->from_id, 1));
    online_.add_user(*opt_uname, msg_ptr->from_id);
  } else {
    send_to_user(msg_ptr->from_id, make_shared<Message>(CommandType::LOGIN, 0, msg_ptr->from_id, 0));
  }
}

\end{lstlisting}

\textbf{server\_main.cpp}

\begin{lstlisting}[language=C++]

#include <signal.h>

#include <filesystem>
#include <iostream>
#include <string>

#include "server.h"

using namespace std;

const int ERR_TERMINATED = 1;

Server* server_ptr = nullptr;
void TerminateByUser(int) {
  if (server_ptr != nullptr) {
    server_ptr->~Server();
    server_ptr->log("Server is terminated by user");
  }
  exit(0);
}

void parse_cmd(Server& server, shared_ptr<Message> msg_ptr) {
  switch (msg_ptr->command) {
    case CommandType::CONNECT:
      server.add_connection(msg_ptr->from_id);
      break;
    case CommandType::DISCONNECT:
      server.remove_connection(msg_ptr->from_id);
      break;
    case CommandType::TEXT:
      if (msg_ptr->type() != MessageType::TEXT) {
        server.log("Text command in non text message");
        break;
      }
      server.send_from_user_to_user(msg_ptr->from_id, msg_ptr);
      break;
    case CommandType::REGISTER:
      if (msg_ptr->type() != MessageType::TEXT) {
        server.log("Register command in non text message");
        break;
      }
      server.register_form(msg_ptr);
      break;
    case CommandType::LOGIN:
      if (msg_ptr->type() != MessageType::TEXT) {
        server.log("Login command in non text message");
        break;
      }
      server.login_form(msg_ptr);
      break;
    case CommandType::CREATE_CHAT:
      if (msg_ptr->type() != MessageType::TEXT) {
        server.log("Register command in non text message");
        break;
      }
      server.create_room(msg_ptr->from_id, ((TextMessage*)msg_ptr.get())->text);
      break;
    case CommandType::LEFT_CHAT:
      server.exit_room(msg_ptr->from_id);
      break;
    case CommandType::FILE_NAME:
      if (msg_ptr->type() != MessageType::TEXT) {
        server.log("FileName command in non text message");
        break;
      }
      server.send_from_user_to_user(msg_ptr->from_id, msg_ptr);
      break;
    case CommandType::FILE_PART:
      if (msg_ptr->type() != MessageType::FILE) {
        server.log("FilePart command in non file message");
        break;
      }
      server.send_from_user_to_user(msg_ptr->from_id, msg_ptr);
      break;
    default:
      throw logic_error("Unimplemented command type");
  }
}

int main() {
  try {
    if (signal(SIGINT, TerminateByUser) == SIG_ERR) {
      throw runtime_error("Can't set SIGINT signal");
    }
    if (signal(SIGSEGV, TerminateByUser) == SIG_ERR) {
      throw runtime_error("Can't set SIGSEGV signal");
    }

    Server server;
    server_ptr = &server;
    server.log("Server is started correctly");

    while (true) {
      shared_ptr<Message> msg_ptr = server.receive();
      parse_cmd(server, msg_ptr);
    }
  } catch (exception& ex) {
    cerr << ("Server exception: "s + ex.what() + "\nServer is terminated by exception");
  }
  return ERR_TERMINATED;
}


\end{lstlisting}

\pagebreak